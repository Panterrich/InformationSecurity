\documentclass[10pt]{beamer}

\usepackage{paratype}
\RequirePackage{cmap}
\RequirePackage[T2A]{fontenc}
\RequirePackage[utf8]{inputenc}

% Set up fonts
%\RequirePackage{paratype} % Arial-like sans serif font
%\RequirePackage{DejaVuSansMono} % Free widespread monospace font with cyrillic support
%\RequirePackage{nimbussans} % Helvetica clone
\RequirePackage[bold,light]{nimbusmono} % Courier clone
%\RequirePackage{tempora}  % "Times" font

\RequirePackage[english,russian]{babel}

\usepackage{amsmath,mathrsfs,amsfonts,amssymb, mathtools}
\usepackage{graphicx, epsfig}
\usepackage{subfig}
\usepackage{setspace}
\usepackage{multirow}
\captionsetup{labelformat=empty}
\usepackage{wrapfig}
\usepackage{array}
\usepackage{multirow}
\usepackage{listings}
\usepackage{color}

\definecolor{dkgreen}{rgb}{0,0.6,0}
\definecolor{gray}{rgb}{0.5,0.5,0.5}
\definecolor{mauve}{rgb}{0.58,0,0.82}

\lstset{frame=tb,
  language=Java,
  aboveskip=3mm,
  belowskip=3mm,
  showstringspaces=false,
  columns=flexible,
  basicstyle={\small\ttfamily},
  numbers=none,
  numberstyle=\tiny\color{gray},
  keywordstyle=\color{blue},
  commentstyle=\color{dkgreen},
  stringstyle=\color{mauve},
  breaklines=true,
  breakatwhitespace=true,
  tabsize=3
}

\usepackage{color, colortbl}
\definecolor{lightRed}{RGB}{240, 170, 150}
\definecolor{lightGreen}{RGB}{170, 230, 150}
\definecolor{lightYellow}{RGB}{240, 230, 180}

\usepackage{changepage}

\setbeamerfont{frametitle}{size=\normalsize,family=\sffamily}
\usetheme{Warsaw}%{Singapore}%{Warsaw}%{Warsaw}%{Darmstadt}
\usecolortheme{sidebartab}
\setbeamertemplate{footline}[author]
\expandafter\def\expandafter\insertshorttitle\expandafter{%
	\insertshorttitle\hfill%
	\insertframenumber\,/\,\inserttotalframenumber}

\definecolor{beamer@blendedblue}{RGB}{3,91,170}
\setbeamercolor{color1}{bg=beamer@blendedblue,fg=white}
% отключить клавиши навигации
\setbeamertemplate{navigation symbols}{}

\usepackage{appendixnumberbeamer}
\usepackage{booktabs}
\usepackage[scale=2]{ccicons}

\usepackage{pgfplots}
\usepgfplotslibrary{dateplot}

\usepackage{xspace}
\newcommand{\themename}{\textbf{\textsc{metropolis}}\xspace}

\usepackage{listings}
\usepackage{xcolor}

\definecolor{codegreen}{rgb}{0,0.6,0}
\definecolor{codegray}{rgb}{0.5,0.5,0.5}
\definecolor{codepurple}{rgb}{0.58,0,0.82}

\lstdefinestyle{gostyle}{
    basicstyle=\ttfamily\footnotesize,
    breakatwhitespace=false,
    breaklines=true,
    captionpos=b,
    keepspaces=true,
    numbers=left,
    numbersep=5pt,
    showspaces=false,
    showstringspaces=false,
    showtabs=false,
    tabsize=2,
    commentstyle=\color{codegreen},
    keywordstyle=\color{magenta},
    stringstyle=\color{codepurple},
    frame=single
}

\usepackage{svg}
\usepackage{todonotes}

\AtBeginSection[]{}

\setbeamertemplate{section in toc}{%
  \alert{$\bullet$}~\inserttocsection}

\graphicspath{{images/}{images2/}{../images/}}

\usepackage{makecell}
\usepackage{changepage}

\title[]{Go. Memory-safe language.}
%\subtitle{Подназвание доклада}

\author{Дурнов Алексей Николаевич}

% Russian
\institute[МФТИ]{
    Московский физико-технический институт\\
    Физтех-школа радиотехники и компьютерных технологий\\
}
\vspace{100pt}
\date{Москва, 2025 г.}

% English
% \institute{ISP RAS}
% \titlegraphic{\hfill\includegraphics[height=0.5cm]{logo_isp_en.png}}

\setbeamercolor{footline}{fg=black}
\setbeamerfont{footline}{series=\bfseries}

\begin{document}

\begin{frame}
    \titlepage
    \thispagestyle{empty}
\end{frame}

\begin{frame}{Cybersecurity and Infrastructure Security Agency (CISA)}
    \begin{block}{}
        The development of new product lines for use in service of critical infrastructure or NCFs in a memory-unsafe language (e.g., C or C++)
        where readily available alternative memory-safe languages could be used is dangerous and significantly elevates risk to national security,
        national economic security, and national public health and safety.
    \end{block}
\end{frame}

\begin{frame}[fragile]
    \frametitle{Безопасная работа с памятью}
    \begin{block}{Автоматическое выделение памяти}
    \begin{lstlisting}[style=gostyle, language=Golang]
func EscapeAnalysis() *int {
    // Local var "escape" into heap
    x := 42
    return &x
}

func StackAllocation() int {
    // Stay in stack
    y := 100
    return y
}
    \end{lstlisting}
    \end{block}
    \begin{itemize}
        \item Компилятор сам решает где размещать данные
        \item Нет ручного управления (new/delete)
    \end{itemize}
    \end{frame}

\begin{frame}[fragile]{}
    \frametitle{Защита от переполнения буфера}
    \begin{block}{Проверка границ слайсов}
    \begin{lstlisting}[style=gostyle, language=Golang]
func CheckBounds() {
    data := []int{1, 2, 3}

    value := data[5] // panic
}
    \end{lstlisting}
    \end{block}
    \begin{itemize}
        \item Автоматическая проверка границ во время выполнения
        \item Паника вместо неопределенного поведения
        \item Возможность восстановления через recover()
    \end{itemize}
\end{frame}

\begin{frame}[fragile]
    \frametitle{Защита от утечек ресурсов}
    \begin{block}{Контекст с таймаутом}
    \begin{lstlisting}[style=gostyle, language=Golang]
func SafeResourceHandling() {
    ctx, cancel := context.WithTimeout(
        context.Background(),
        2*time.Second,
    )
    defer cancel()

    req, _ := http.NewRequestWithContext(
        ctx,
        "GET",
        "https://example.com",
        nil,
    )

    resp, _ := http.DefaultClient.Do(req)
    defer resp.Body.Close()

    // Response processing
}
    \end{lstlisting}
    \end{block}
    \end{frame}

    \begin{frame}[fragile]
        \frametitle{Конкурентность из "коробки"}
        \begin{block}{Синхронизация горутин}
        \begin{lstlisting}[style=gostyle, language=Golang]
func SafeConcurrency() {
    var wg sync.WaitGroup
    sharedMap := make(map[int]string)
    var mu sync.Mutex

    for i := 0; i < 10; i++ {
        wg.Add(1)
        go func(idx int) {
            defer wg.Done()
            mu.Lock()
            sharedMap[idx] = fmt.Sprintf("Data %d", idx)
            mu.Unlock()
        }(i)
    }
    wg.Wait()
}
        \end{lstlisting}
        \end{block}
        \end{frame}

        \frametitle{Уязвимости при работе с CGO}
        \begin{block}{Прямой доступ к памяти через C}
        \begin{lstlisting}[style=gostyle, language=Golang]
/*
#include <stdlib.h>
*/
import "C"
import "unsafe"

func unsafeAlloc() {
    ptr := C.malloc(1024)
    // Memory leak without free
    // C.free(unsafe.Pointer(ptr))

    // Unsafe conversion
    data := (*byte)(unsafe.Pointer(ptr))
    *data = 255
}
        \end{lstlisting}
        \end{block}
        \begin{itemize}
            \item Прямой доступ к сырой памяти через unsafe
            \item Риск утечек и повреждения памяти
            \item Потеря всех гарантий безопасности Go
        \end{itemize}
        \end{frame}

        \begin{frame}[fragile]
        \frametitle{Утечка горутин}
        \begin{block}{Необработанные задачи}
        \begin{lstlisting}[style=gostyle, language=Golang]
func WorkerPoolLeak() {
    jobs := make(chan int, 100)

    // Start workers
    for i := 0; i < 5; i++ {
        go func() {
            for job := range jobs {
                process(job)
            }
        }()
    }

    for i := 0; i < 10; i++ {
        jobs <- i
    }

    // Forgot to close the jobs channel
    // The workers continue to wait for the task
}
        \end{lstlisting}
        \end{block}
        \end{frame}

\end{document}
