\documentclass[10pt]{beamer}

\usepackage{paratype}
\RequirePackage{cmap}
\RequirePackage[T2A]{fontenc}
\RequirePackage[utf8]{inputenc}

% Set up fonts
%\RequirePackage{paratype} % Arial-like sans serif font
%\RequirePackage{DejaVuSansMono} % Free widespread monospace font with cyrillic support
%\RequirePackage{nimbussans} % Helvetica clone
\RequirePackage[bold,light]{nimbusmono} % Courier clone
%\RequirePackage{tempora}  % "Times" font

\RequirePackage[english,russian]{babel}

\usepackage{amsmath,mathrsfs,amsfonts,amssymb, mathtools}
\usepackage{graphicx, epsfig}
\usepackage{subfig}
\usepackage{setspace}
\usepackage{multirow}
\captionsetup{labelformat=empty}
\usepackage{wrapfig}
\usepackage{array}
\usepackage{multirow}
\usepackage{listings}
\usepackage{color}

\definecolor{dkgreen}{rgb}{0,0.6,0}
\definecolor{gray}{rgb}{0.5,0.5,0.5}
\definecolor{mauve}{rgb}{0.58,0,0.82}

\lstset{frame=tb,
  language=Java,
  aboveskip=3mm,
  belowskip=3mm,
  showstringspaces=false,
  columns=flexible,
  basicstyle={\small\ttfamily},
  numbers=none,
  numberstyle=\tiny\color{gray},
  keywordstyle=\color{blue},
  commentstyle=\color{dkgreen},
  stringstyle=\color{mauve},
  breaklines=true,
  breakatwhitespace=true,
  tabsize=3
}

\usepackage{color, colortbl}
\definecolor{lightRed}{RGB}{240, 170, 150}
\definecolor{lightGreen}{RGB}{170, 230, 150}
\definecolor{lightYellow}{RGB}{240, 230, 180}

\usepackage{changepage}

\setbeamerfont{frametitle}{size=\normalsize,family=\sffamily}
\usetheme{Warsaw}%{Singapore}%{Warsaw}%{Warsaw}%{Darmstadt}
\usecolortheme{sidebartab}
\setbeamertemplate{footline}[author]
\expandafter\def\expandafter\insertshorttitle\expandafter{%
	\insertshorttitle\hfill%
	\insertframenumber\,/\,\inserttotalframenumber}

\definecolor{beamer@blendedblue}{RGB}{3,91,170}
\setbeamercolor{color1}{bg=beamer@blendedblue,fg=white}
% отключить клавиши навигации
\setbeamertemplate{navigation symbols}{}

\usepackage{appendixnumberbeamer}
\usepackage{booktabs}
\usepackage[scale=2]{ccicons}

\usepackage{pgfplots}
\usepgfplotslibrary{dateplot}

\usepackage{xspace}
\newcommand{\themename}{\textbf{\textsc{metropolis}}\xspace}

\usepackage{svg}
\usepackage{todonotes}

\AtBeginSection[]{}

\setbeamertemplate{section in toc}{%
  \alert{$\bullet$}~\inserttocsection}

\graphicspath{{images/}{images2/}{../images/}}

\usepackage{makecell}
\usepackage{changepage}

\title[]{Задание по RBAC}
%\subtitle{Подназвание доклада}

\author{Дурнов Алексей Николаевич}

% Russian
\institute[МФТИ]{
    Московский физико-технический институт\\
    Физтех-школа радиотехники и компьютерных технологий\\
}
\vspace{100pt}
\date{Москва, 2025 г.}

% English
% \institute{ISP RAS}
% \titlegraphic{\hfill\includegraphics[height=0.5cm]{logo_isp_en.png}}

\setbeamercolor{footline}{fg=black}
\setbeamerfont{footline}{series=\bfseries}

\begin{document}

\begin{frame}
    \titlepage
    \thispagestyle{empty}
\end{frame}

\begin{frame}{Состав персонала}
    \begin{enumerate}
        \item Руководство и менеджмент (CEO, CTO, COO, PM)
        \item Разработка (TechLead, Senior, Middle, Junior, QA, DevOps, UI/UX)
        \item Поддержка и инфраструктура (Сис. админ, Support)
        \item Отделы бизнеса и маркетинга (Product, Sales, Marketing Managers)
        \item Финансовый и административный блок (CFO, HR, Office Manager)
        \item Дополнительные роли (Data analyst, Legal Advisor)
    \end{enumerate}
\end{frame}

\begin{frame}{Активы компании}
    Выберем основной набор активов обычной IT-компании:
    \begin{itemize}
            \item Офис
            \item Серверная
            \item Корпоративный мессенджер
            \item Корпоративная почта
            \item Корпоративный VPN
            \item Корпоративный портал поддержки
            \item Система мониторинга (Zabbix)
            \item База знаний (Confluence)
            \item Репозиторий (Microsoft Azure)
            \item CI/CD (Hive)
            \item Инструменты QA (Asgard)
            \item Сервис для моделирование макетов (Figma)
            \item Трекер задач и требований (Microsoft Azure)
            \item CRM-система (Customer Relationship Management)
            \item ERP-система (Enterprise Risk Management)
        \end{itemize}
\end{frame}

\begin{frame}{Разрешения (P)}
    Введём для выбранных активов основные разрешения в общем виде:
    \begin{itemize}
        \item Офис: доступ в офис в любое время, доступ к оборудованию (принтерам/сканерам).
        \item Серверная: доступ в серверную.
        \item Корпоративный мессенджер: аккаунт с доступом к общим чатам команд и возможность написать каждому сотруднику.
        \item Корпоративная почта: личный ящик (отправка/получение), доступ к общим папкам, регистрация в корпоративных сервисах.
        \item Корпоративный VPN: удаленный доступ к корпоративной сети.
        \item Корпоративный портал поддержки: создание тикетов, просмотр истории обращений.
        \item Система мониторинга: просмотр метрик, изменение метрик и настройка оповещений.
        \item База знаний: просмотр/редактирование страниц.
        \item Репозиторий: просмотр кода, отправление коммитов, создание веток и Pull Request'ов.
    \end{itemize}
\end{frame}

\begin{frame}{Разрешения (P)}
    \begin{itemize}
        \item CI/CD: запуск сборок и изменение конфигурации pipeline.
        \item Инструменты QA: редактирование и запуск тестов, просмотр отчетов.
        \item Сервис для моделирование макетов (Figma): просмотр макетов, редактирование и создание компонентов.
        \item Трекер задач и требований: создание задач и требований, изменение статусов, назначение исполнителей.
        \item CRM-система: просмотр контактов клиентов, редактирование сделок, генерация отчетов.
        \item ERP-система: ввод данных, выполнение финансовых операций, просмотр аналитики.
    \end{itemize}
\end{frame}

\begin{frame}{Роли (R)}
    Определим роли, исходя из классов эквивалентности по разрешениям. Но перед этим определим общие разрешения, которые будут у каждой роли (иерархия ролей: роль - сотрудник):
    \begin{itemize}
        \item Доступ в офис в любое время, доступ к оборудованию (принтерам/сканерам).
        \item Аккаунт в мессенджере с доступом к общим чатам команд и возможность написать каждому сотруднику.
        \item Личный ящик (отправка/получение), доступ к общим папкам, регистрация в корпоративных сервисах.
        \item Доступ к корпоративному VPN.
        \item Cоздание тикетов и просмотр истории обращений на внутреннем портале поддержки.
    \end{itemize}
\end{frame}

\begin{frame}{Роли (R, UA + PA)}
    \begin{itemize}
        \item Officer (CEO, CTO, COO) - доступ к аналитике CRM/ERP, трекеру задач и требований, доступ к базе знаний.
        \item Project Manager (PM) - трекер задач и требований, доступ к базе знаний.
        \item TechLead - просмотр кода, создание задач и изменение статусов, доступ к базе знаний, изменение конфигурации pipeline'ов, настройка оповещений системы мониторинга.
        \item Developer (Senior, Middle, Junior) - просмотр кода, отправление коммитов, создание веток и Pull Request'ов, создание задач и изменение статусов, доступ к базе знаний, запуск pipeline'ов
        \item QA Engineer - доступ в серверную, редактирование и запуск тестов, просмотр кода, отправление коммитов, создание веток и Pull Request'ов, создание баг-задач, доступ к базе знаний.
        \item DevOps Engineer - доступ в серверную, изменение конфигурации pipeline'ов, настройка системы мониторинга, создание задач и изменение статусов.
        \item UI/UX Designer - просмотр макетов, редактирование и создание компонентов, создание задач и изменение статусов.
    \end{itemize}
\end{frame}

\begin{frame}{Роли (R, UA + PA)}
    \begin{itemize}
        \item Сис. админ - доступ в серверную, настройка VPN и других корпоративных сервисов.
        \item Support - закрытие тикетов на портале поддержки.
        \item Business Manager (Product, Sales, Marketing Managers) - доступ к аналитике CRM/ERP, редактирование данных в CRM.
        \item CFO - доступ к аналитике CRM/ERP, редактирование данных в ERP.
        \item HR - доступ к базе знаний (HR-документов).
        \item Office Manager - управление оборудованием, ввод данных по закупкам в ERP.
        \item Data Analyst - доступ к аналитике CRM/ERP, просмотр метрик из системы мониторинга.
        \item Legal Advisor - доступ к базе знаний (юридические документы).
    \end{itemize}
\end{frame}

\begin{frame}{Ограничения}
    \begin{itemize}
        \item Взаимное исключение ролей:
        \begin{itemize}
            \item Сотрудник не может иметь одновременно две роли QA Engineer и Developer
            \item Сотрудник не может иметь одновременно две роли CFO и Office Manager
            \item Сотрудник не может иметь техническую роль и менеджерскую роль.
        \end{itemize}
        \item Количественное ограничение ролей:
        \begin{itemize}
            \item Не более 3 сотрудников в роли Officer.
            \item Минимум 1 сотрудник в роли DevOps.
            \item Сотрудников с ролью QA не меньше, чем в 5 раз, чем Developer'ов.
        \end{itemize}
    \end{itemize}
\end{frame}

\begin{frame}{Административные роли (AP, AR, APA, AUA)}
    \begin{itemize}
        \item Role Administrator (CEO) - назначать/отзывать роли пользователям.
        \begin{itemize}
            \item can-assign:
            \begin{itemize}
                \item Officer: сотрудник должен иметь стаж в компании ≥ 5 лет и подтверждение от совета директоров.
                \item Project Manager: пользователь должен получить рекомендацию хотя бы от одного Officer.
                \item TechLead: сотрудник должен иметь роль Developer ≥ 5 лет.
                \item DevOps: пользователь должен пройти сертификацию по облачным технологиям.
                \item Business Manager: пользователь должен быть утвержден текущим CFO.
                \item Остальные роли: пользователь должен успешно пройти интервью (получить 2 рекомендации из 3).
            \end{itemize}
            \item can-revoke:
            \begin{itemize}
                \item Officer: требуется согласие двух других Officer или совета директоров.
                \item Остальные роли: уведомление от HR/выше стоящего менеджера и подтверждение причины отзыва роли.
            \end{itemize}
        \end{itemize}
    \end{itemize}
\end{frame}

\begin{frame}{Административные роли (AP, AR, APA, AUA)}
    \begin{itemize}
        \item Policy Administrator (COO) - определять и настраивать статические и динамические ограничения для ролей RBAC.
            \begin{itemize}
                \item не участвует в назначении ролей и разрешений.
            \end{itemize}
        \item Service Administrator (Support) - предоставлять доступы к необходимым сервисам сотрудникам.
        \begin{itemize}
            \item can-assign и can-revoke:
            \begin{itemize}
                \item Все роли: при доставление доступа к указанным выше сервисам у соответствующей роли.
            \end{itemize}
        \end{itemize}
        \item Security Administrator (Support) - блокировка пользователей при подозрительной активности.
        \begin{itemize}
            \item can-revoke:
            \begin{itemize}
                \item Все роли: при подозрительной активности пользователя отзывается роль до окончательного вынесения решения.
            \end{itemize}
        \end{itemize}
    \end{itemize}
\end{frame}

\end{document}
