\documentclass[10pt]{beamer}

\usepackage{paratype}
\RequirePackage{cmap}
\RequirePackage[T2A]{fontenc}
\RequirePackage[utf8]{inputenc}

% Set up fonts
%\RequirePackage{paratype} % Arial-like sans serif font
%\RequirePackage{DejaVuSansMono} % Free widespread monospace font with cyrillic support
%\RequirePackage{nimbussans} % Helvetica clone
\RequirePackage[bold,light]{nimbusmono} % Courier clone
%\RequirePackage{tempora}  % "Times" font

\RequirePackage[english,russian]{babel}

\usepackage{amsmath,mathrsfs,amsfonts,amssymb, mathtools}
\usepackage{graphicx, epsfig}
\usepackage{subfig}
\usepackage{setspace}
\usepackage{multirow}
\captionsetup{labelformat=empty}
\usepackage{wrapfig}
\usepackage{array}
\usepackage{multirow}
\usepackage{listings}
\usepackage{color}

\definecolor{dkgreen}{rgb}{0,0.6,0}
\definecolor{gray}{rgb}{0.5,0.5,0.5}
\definecolor{mauve}{rgb}{0.58,0,0.82}

\lstset{frame=tb,
  language=Java,
  aboveskip=3mm,
  belowskip=3mm,
  showstringspaces=false,
  columns=flexible,
  basicstyle={\small\ttfamily},
  numbers=none,
  numberstyle=\tiny\color{gray},
  keywordstyle=\color{blue},
  commentstyle=\color{dkgreen},
  stringstyle=\color{mauve},
  breaklines=true,
  breakatwhitespace=true,
  tabsize=3
}

\usepackage{color, colortbl}
\definecolor{lightRed}{RGB}{240, 170, 150}
\definecolor{lightGreen}{RGB}{170, 230, 150}
\definecolor{lightYellow}{RGB}{240, 230, 180}

\usepackage{changepage}

\setbeamerfont{frametitle}{size=\normalsize,family=\sffamily}
\usetheme{Warsaw}%{Singapore}%{Warsaw}%{Warsaw}%{Darmstadt}
\usecolortheme{sidebartab}
\setbeamertemplate{footline}[author]
\expandafter\def\expandafter\insertshorttitle\expandafter{%
	\insertshorttitle\hfill%
	\insertframenumber\,/\,\inserttotalframenumber}

\definecolor{beamer@blendedblue}{RGB}{3,91,170}
\setbeamercolor{color1}{bg=beamer@blendedblue,fg=white}
% отключить клавиши навигации
\setbeamertemplate{navigation symbols}{}

\usepackage{appendixnumberbeamer}
\usepackage{booktabs}
\usepackage[scale=2]{ccicons}

\usepackage{pgfplots}
\usepgfplotslibrary{dateplot}

\usepackage{xspace}
\newcommand{\themename}{\textbf{\textsc{metropolis}}\xspace}

\usepackage{svg}
\usepackage{todonotes}

\AtBeginSection[]{}

\setbeamertemplate{section in toc}{%
  \alert{$\bullet$}~\inserttocsection}

\graphicspath{{images/}{images2/}{../images/}}

\usepackage{makecell}
\usepackage{changepage}

\title[]{Задание по информационной безопасности}
%\subtitle{Подназвание доклада}

\author{Дурнов Алексей Николаевич}

% Russian
\institute[МФТИ]{
    Московский физико-технический институт\\
    Физтех-школа радиотехники и компьютерных технологий\\
}
\vspace{100pt}
\date{Москва, 2024 г.}

% English
% \institute{ISP RAS}
% \titlegraphic{\hfill\includegraphics[height=0.5cm]{logo_isp_en.png}}

\setbeamercolor{footline}{fg=black}
\setbeamerfont{footline}{series=\bfseries}

\begin{document}

\begin{frame}
    \titlepage
    \thispagestyle{empty}
\end{frame}

\begin{frame}{Цели организации в области создания безопасного ПО}
    Основными целями организации в области создания безопасного ПО являются:

    \begin{enumerate}
        \item Минимизация рисков, связанных с различными уязвимостями ПО,
        \item Соответствие стандартам и требованиям безопасности,
        \item Повышение качества кода,
        \item Оптимизация затрат на поддержку и эксплуатацию продукта,
        \item Улучшение репутации компании.
    \end{enumerate}

\end{frame}

\begin{frame}{Перечень и описание мер по разработке безопасного ПО, подлежащих реализации в среде разработки ПО}
    \begin{itemize}
        \item Проектирование и архитектура
        \begin{enumerate}
            \item Разделяйте код на независимые модули, каждый из которых отвечает за свою конкретную задачу. Это сокращает время поиска и устранения уязвимостей.
            \item Проектируйте приложение таким образом, чтобы оно работало с минимально возможными правами доступа. Это уменьшает возможные последствия успешных атак.
            \item Определите и документируйте архитектуру безопасности вашего приложения, включая модели угроз и методы защиты.
        \end{enumerate}
    \end{itemize}
\end{frame}


\begin{frame}{Перечень и описание мер по разработке безопасного ПО, подлежащих реализации в среде разработки ПО}
    \begin{itemize}
        \item Написание кода
        \begin{enumerate}
            \item Работайте с последними стандартами языка, которые предоставляют множество возможностей для безопасного программирования.
            \item Используйте по возможности умные указатели для автоматического управление памятью, чтобы избегать утечек памяти и двойного освобождения.
            \item Применяйте идиому RAII (Resource Acquisition Is Initialization) для автоматического управления ресурсами, включая файлы, сокеты и другие объекты. Не работайте с примитивами синхронизации (std::mutex и другие) без RAII оберток.
            \item Используйте std::string вместо сырых строк типа char*. Это предотвращает многие проблемы, связанные с переполнением буфера.
            \item Всегда экранируйте и валидируйте пользовательские данные, особенно если они используются в SQL-запросах, HTML-коде и т.д.
        \end{enumerate}
    \end{itemize}

\end{frame}

\begin{frame}{Перечень и описание мер по разработке безопасного ПО, подлежащих реализации в среде разработки ПО}
    \begin{itemize}
    \item Тестирование и анализ кода
    \begin{enumerate}
        \item Проводите юнит-тестирование и функциональное тестирования для проверки корректности работы отдельных функций и модулей.
        \item Используйте инструменты статического анализа (например, cppcheck, svace, pvs-studio) для выявления потенциальных уязвимостей и ошибок.
        \item Используйте инструменты динамического анализа (например, valgrind, clang code sanitizers) для обнаружения уязвимостей и ошибок.
        \item Проводите фаззинг-тестирования для поиска сбоев программы.
        \item Проводите тестирование на проникновение для выявления уязвимостей в реальных условиях эксплуатации.
    \end{enumerate}

    \item Управление уязвимостями
    \begin{enumerate}
        \item Быстро реагируйте на обнаруженные уязвимости, выпуская патчи и обновления.
        \item Следите за новыми уязвимостями в используемых библиотеках и компонентах. Регулярно обновляйте зависимости.
        \item Ведите журнал всех обнаруженных уязвимостей и их статуса (исправлено, отложено и т.д.).
    \end{enumerate}
\end{itemize}
\end{frame}


\begin{frame}{Перечень и описание мер по разработке безопасного ПО, подлежащих реализации в среде разработки ПО}
    \begin{itemize}
\item Документирование и обучение
\begin{enumerate}
    \item Создавайте и поддерживайте документацию по безопасности, включая описания политик, процедур и рекомендуемых практик.
    \item Регулярно обучайте разработчиков основам безопасного программирования и новым технологиям.
    \item Периодически проводите внешние аудиты безопасности для независимой оценки качества безопасности вашего ПО.
\end{enumerate}

\item Инфраструктура
\begin{enumerate}
    \item Внедряйте логирование и мониторинг для отслеживания аномального поведения и попыток взлома.
    \item Регулярно делайте резервные копии данных и тестируйте процедуры восстановления.
\end{enumerate}

\item Сопровождение и поддержка
\begin{enumerate}
    \item Регулярно обновляйте операционную систему, библиотеки и другие компоненты для устранения известных уязвимостей.
    \item Обучайте пользователей правилам безопасного использования вашего ПО и помогайте им решать проблемы, связанные с безопасностью.
\end{enumerate}
\end{itemize}
\end{frame}

\begin{frame}{Перечень документации разработчика ПО, связанной с реализацией мер по разработке безопасного ПО}
    \begin{enumerate}
        \item Описания технических требований к проекту.
        \item Кодекс безопасности: внутренний документ, содержащий лучшие практики и стандарты по безопасной разработке ПО.
        \item Отчеты по тестированию: результаты статических и динамических анализов кода, отчеты по фаззинг-тестированию, отчеты по тестированию на проникновение.
        \item Журналы аудита: записи о проведенных аудитах безопасности и результаты проверок.
        \item Политика информационной безопасности: документ описывает общие принципы и правила работы с информацией внутри организации, включая меры по защите данных, доступу к ресурсам и управлению инцидентами.
    \end{enumerate}
\end{frame}

\begin{frame}{Описание действий, направленных на улучшение процессов, связанных с разработкой безопасного ПО}
    \begin{enumerate}
        \item \textbf{Интеграция принципов безопасной разработки в культуру компании}.

        Проводите регулярные тренинги и семинары для разработчиков по вопросам информационной безопасности. Внедрите систему мотивации для разработчиков, которые активно участвуют в улучшении безопасности кода.

        \item \textbf{Автоматизация процессов}.

        Интегрируйте инструменты статического анализа кода и тестирования безопасности в систему непрерывной интеграции.
        Настройте автоматическую проверку и обновление используемых библиотек и компонентов на наличие последних версий, содержащих исправления безопасности. Внедрите системы мониторинга и оповещения, которые будут уведомлять разработчиков и администраторов безопасности о возникновении новых уязвимостей или инцидентов.

        \item \textbf{Применение лучших практик программирования}.

        Переходите на использование современных стандартов C++. Современные версии языка содержат встроенные механизмы защиты от уязвимостей.

    \end{enumerate}
\end{frame}

\begin{frame}{<<JSON for Modern C++>>}

    \begin{itemize}
        \item Библиотека <<nlohmann/json>> (также известная как <<JSON for Modern C++>>) представляет собой высокоэффективную библиотеку для работы с JSON-форматом данных в языке программирования C++. Она позволяет легко сериализовать и десериализовать объекты в формат JSON и обратно.

        \item Основные особенности проекта:
        \begin{enumerate}
            \item \textbf{Интуитивно понятный синтаксис}: с помощью перегрузки операторов удалось добиться удобного и интуитивно понятного интерфейса на ровне с другими высокоуровневыми языками.
            \item \textbf{Простота интеграции}: весь код состоит из одного заголовочного файла <<json.hpp>>. У проекта нет ни внешних зависимостей, ни сложной системы сборки. Библиотека написана на C++11.
            \item \textbf{Высокая производительность}: библиотека оптимизирована для высокой производительности при работе с большими объемами данных, несмотря на простоту интерфейса.
           \item \textbf{Открытый исходный код}: проект имеет лицензию MIT, что делает его доступным для свободного использования в коммерческих и некоммерческих проектах.
        \end{enumerate}
    \end{itemize}

\end{frame}

\begin{frame}{Требования по безопасности, предъявляемые к разрабатываемому ПО}
    \begin{itemize}
        \item Проверка входных данных: неправильный формат данных и внедрение вредоносного кода.

        \item Защита от атак типа "отказ в обслуживании" (DoS): необходимо контролировать объем и сложность входных данных, устанавливая лимиты на количество уровней вложенности, длину строк и общее число элементов.

        \item Процесс преобразования объектов C++ в JSON и обратно должен быть защищен от возможных уязвимостей. \textbf{Десериализованные} объекты могут содержать конфиденциальную информацию, поэтому важно убедиться, что доступ к ним ограничен только авторизованным пользователям.

        \item Изоляция среды выполнения: среда выполнения не подвержена утечкам информации или несанкционированному доступу к ресурсам.

        \item Аудит кода и тестирование.
    \end{itemize}
\end{frame}

\begin{frame}{Сведения о результатах моделирования угроз безопасности информации}
    \begin{itemize}
        \item Для моделировании угроз безопасности информации будет использоваться формальная модель STRIDE.

        \item Модель будет охватывать саму библиотеку и её взаимодействие с данными, которые она обрабатывает.
        \item Защищаемыми активами являются исходный код библиотеки, данные, обрабатываемые библиотекой, инфраструктура разработки и среда исполнения.

    \end{itemize}

    Классификация угроз по модели STRIDE:
    \begin{itemize}
        \item Подмена личности (Spoofing)
        \begin{enumerate}
            \item Утечка учетных данных разработчиков: если учетные данные разработчиков будут украдены, злоумышленники смогут выдавать себя за легитимных участников проекта и вносить изменения в исходный код.
        \end{enumerate}

        \item Несанкционированное изменение (Tampering)
        \begin{enumerate}
            \item Изменение исходного кода: злоумышленники могут попытаться внести изменения в исходный код библиотеки, добавив вредоносный код или удалив важные элементы безопасности.
        \end{enumerate}
    \end{itemize}

\end{frame}


\begin{frame}{Сведения о результатах моделирования угроз безопасности информации}
    \begin{itemize}
        \item Отказ от ответственности (Repudiation)
        \begin{enumerate}
            \item Отсутствие аудита: отсутствие должного аудита и журналирования действий может затруднить расследование инцидентов и установление виновных.
            \item Невозможность доказать подлинность изменений: без надлежащей системы контроля версий и управления изменениями невозможно точно установить, кто и когда вносил изменения в код.
        \end{enumerate}

        \item Раскрытие информации ( Information Disclosure)
        \begin{enumerate}
            \item Некорректная обработка данных: Некорректная обработка JSON-данных может привести к утечке конфиденциальной информации.
            \item SQL-инъекции и XSS (Cross-Site Scripting): обработки запросов к базам данных или отображение данных в веб-интерфейсе может привести к раскрытию информации.
        \end{enumerate}

        \item Отказ в обслуживании ( Denial of Service)
        \begin{enumerate}
            \item Переполнение буфера и утечки памяти: может привести к исчерпанию ресурсов и отказу в обслуживании.
        \end{enumerate}

        \item Повышение привилегий (Elevation of Privilege)
        \begin{enumerate}
            \item Неправильное управление правами доступа: недостаточное разграничение прав доступа может позволить злоумышленникам получить больше полномочий, чем им положено.
        \end{enumerate}
    \end{itemize}
\end{frame}

\begin{frame}{Сведения о проекте архитектуры программы}
    Архитектурно проект организован следующим образом:

    \begin{enumerate}
        \item \textbf{Центральным классом} данной библиотеки является класс json: основной контейнер для хранения и обработки JSON-данных. Данный класс обеспечивает работу с различными типами данных: строками, числами, массивами, булевскими значениями и другими JSON-объектами.

        \item Библиотека предлагает удобные механизмы для автоматической \textbf{сериализации} и \textbf{десериализации} стандартных типов данных C++.
            Также поддерживается возможность расширения функциональности для пользовательских типов через перегрузку специальных функций.
        \item Поставляется в виде \textbf{одного заголовочного файла}. Данный вид поставки библиотеки упрощает интеграцию в проекты за счёт отсутствия отдельной компиляции и линковки библиотеки.
        \item В репозитории содержится \textbf{обширная база тестов}. Документация также доступна в репозитории и включает в себя базовые примеры кода, руководство по использованию и справочную информацию по API библиотеки.
    \end{enumerate}
\end{frame}

\begin{frame}{Используемые инструментальные средства}

    Ниже представлен список основных используемых инструментальные средств в проекте:

    \begin{enumerate}
        \item CMake -- основной инструмент сборки проектов. Используется для создания кросс-платформенной среды разработки и сборки библиотеки. С помощью CMake можно генерировать разные системы сборки.
        \item Mkdocs -- инструмент для генерации документации из комментариев в исходном коде. Позволяет создавать HTML, LaTeX и PDF-документы с описанием классов, методов и функций.
        \item GitHub Actions -- система непрерывной интеграции, которая автоматически запускает тесты и проверки кода после изменения в проекте. Это помогает быстро выявлять ошибки и поддерживать высокое качество кода.
        \item Doctest -- один из основных фреймворков для тестирования кода на C++, используемый для написания юнит-тестов.
        \item Clang-format -- инструмент для автоматического форматирования кода.

    \end{enumerate}

\end{frame}

\begin{frame}{Информация о прослеживаемости исходного кода программы к проекту архитектуры программы}

    В данной библиотеке простая структура проекта. Исходный код разбит на нескольких ключевых компонентов:

    \begin{enumerate}
        \item json.hpp -- главный заголовочный файл, содержащий всю функциональность библиотеки. Этот файл включает определения основных классов и функций, таких как класс json, методы для сериализации и десериализации, а также вспомогательные классы и структуры.
        \item detail -- директория, содержащая реализацию внутренней логики библиотеки. Здесь находятся специализированные алгоритмы, утилиты и вспомогательные функции, которые не предназначены для прямого использования пользователями библиотеки.
        \item tests -- директория, содержащая тесты, написанными с использованием фреймворка doctest.
        \item examples -- директория, содержащая примеры использования библиотеки, демонстрирующие базовые сценарии работы с JSON-данными. Эти примеры помогают новым пользователям быстрее освоиться с функционалом библиотеки.
    \end{enumerate}

\end{frame}

\begin{frame}{Порядок оформления исходного кода программы}
    \begin{itemize}
        \item Порядок оформления исходного кода программы определен и закреплен с помощью конфигурационного файла clang-format, находящегося в корне репозитория.
    \end{itemize}
\end{frame}

\begin{frame}{Сведения о результатах проведения экспертизы исходного кода программы}
    \begin{itemize}
        \item Сведений о результатах проведения внешних экспертных аудитов не содержится в репозитории, но можно отметить, что репозиторий имеет более 250 участников разработки (contributer).
        \item Библиотека придерживается рекомендациям безопасной разработки от Core Infrastructure Initiative (CII).
    \end{itemize}
\end{frame}

\begin{frame}{Сведения о результатах проведения функционального тестирования программы}
    \begin{itemize}
        \item Библиотека имеет обширный набор тестов.
        \item Написаны они с использованием фреймворка doctest.
        \item Код библиотеки прошел тщательное модульное тестирование, имеет 100\% покрытие кода, включая все исключительные ситуации.
    \end{itemize}
\end{frame}

\begin{frame}{Сведения о результатах проведения тестирования на проникновение}
    \begin{itemize}
        \item Открытые сведения о результатах проведения тестирования на проникновения у библиотеки отсутствуют.
        \item Однако известно, что данная библиотека прошла проверку в Лаборатории Касперского и Apple.
    \end{itemize}
\end{frame}

\begin{frame}{Сведения о результатах проведения динамического анализа кода программы}
    \begin{itemize}
        \item Для динамического анализа кода программы используются следующие инструменты: valgrind и clang code sanitizers.
        \item Данные инструменты помогают находить такого рода ошибки как утечки памяти, неопределенное поведение, использование неинициализированной памяти, состояние гонок, выходы за граница памяти и другого рода ошибок. В результате применений данных инструментов ошибок подобного рода не было выявлено.

        \item Помимо динамического анализа кода библиотеки используются также и статический анализ: cppcheck и прочие утилиты.
        \item Оба вида анализа дополняют друг друга для обеспечения безопасности библиотеки.

    \end{itemize}
\end{frame}

\begin{frame}{Сведения о результатах проведения фаззинг-тестирования программы}
    \begin{itemize}

    \item Для фаззинг-тестирования в проекте используется Google OSS-Fuzz.
    \item Он дополнительно запускает фаззинг-тесты со всеми анализаторами в режиме реального времени.
    \item На данный момент корпус тестов насчитывает более миллиарда тестов.

    \end{itemize}
\end{frame}

\begin{frame}{Описание процедур отслеживания и исправления обнаруженных ошибок ПО и уязвимостей программы}

    \begin{itemize}
        \item Данная библиотека является с открытым исходным кодом.

        \item Пользователи и разработчики могут сообщать об ошибках и уязвимостях несколькими способами:

        \begin{enumerate}
            \item \textbf{Issue} - основной способ уведомления разработчиков об ошибках и проблемах.
            \item Если проблема \textbf{связана с безопасностью}, то рекомендуется использовать соответствующую функцию Github, чтобы сообщить о проблеме \textbf{приватно}. Это позволит команде разработчиков исправить уязвимость до публичного раскрытия информации.
        \end{enumerate}

        \item После получения отчёта команда разработчиков анализирует серьезность ошибки или уязвимости, оценивает последствия возможных атак для пользователей, устанавливает приоритет задачи в зависимости от ее влияния на стабильность и безопасность библиотеки.

        \item Создается патч, который проходит строгое код-ревью, чтобы гарантировать его корректность и отсутствие новых ошибок, а также проводится полное тестирование, включая регрессионное тестирование, чтобы убедиться, что исправление не вызывает новых проблем.

    \end{itemize}
\end{frame}


\begin{frame}{Описание процедуры поиска разработчиком ПО уязвимостей программы}
    \begin{itemize}
        \item Процедура поиска уязвимостей в программном обеспечении требует систематического подхода и применения различных инструментов и методик.

        \item Основные шаги, которые разработчик может предпринять, для поиска уязвимостей следующие:

        \begin{enumerate}
            \item Статический анализ кода

            Инструменты: cppcheck, clang-tidy, svace, pvs-studio.

            \item Динамический анализ

            Инструменты: valgrind, clang code sanitizers.

            \item Фаззинг

            Инструменты: AFL++, Google OSS-Fuzz.

            \item Тестирование на проникновение

            Инструменты: Burp Suite, OWASP ZAP.

        \end{enumerate}
    \end{itemize}
\end{frame}


\begin{frame}{Реализация и использование процедуры уникальной маркировки каждой версии ПО}
    \begin{itemize}
        \item Используется \textbf{семантическое версионирование} и \textbf{система контроля версий Git}. Версия ПО согласно этому стандарту имеет вид major.minor.patch, где:

        \begin{enumerate}
            \item major: увеличивается при добавлении несовместимых изменений;
            \item minor: увеличивается при добавлении новых функций, не нарушающих совместимость;
            \item patch: увеличивается при исправлении багов, не влияя на совместимость.
        \end{enumerate}

        \item Каждая новая версия библиотеки помечается \textbf{уникальным тегом в системе контроля версий Git}. Также каждой новой версии создается файл со \textbf{списком изменений}, внесенные в данную версию.

        \item Выпуск новой версии состоит из следующих шагов: завершается разработка и тестирования всех изменений, определяется номер новой версии продукта, генерируется пакет для распространения по необходимости, оформляется список изменений, и затем эти изменения публикуются на Github с тегом новой версии.

    \end{itemize}

    \end{frame}


\begin{frame}{Использование системы управления конфигурацией ПО}
    \begin{itemize}
        \item  Система управления конфигурацией охватывает \textbf{систему контроля версий, построения и распространения}.

        \item Для данной библиотеки используется \textbf{система контроля версий Git}. Git позволяет разработчикам отслеживать изменения в кодовой базе, создавать ветки для параллельной разработки, объединять изменения и фиксировать состояние кода на определенных этапах.

        \item Для автоматизации процесса сборки и тестирования используется \textbf{система непрерывной интеграции, основанная на GitHub Actions}. Эта система позволяет автоматически запускать сборку и тестирование кода при каждом изменении в репозитории.
    \end{itemize}

\end{frame}

\begin{frame}{Меры, используемые для защиты инфраструктуры среды разработки ПО}
    В данном проекте используются следующие меры:

    \begin{enumerate}
        \item \textbf{Управление доступом} на Github происходит с помощью ограничений прав членов команды. Разработчики имеют разные уровни доступа в зависимости от своей роли (читатель, участник, администратор). Это помогает предотвращать случайные или злонамеренные действия со стороны отдельных участников.
        \item \textbf{Two-Factor Authentication (2FA)}: рекомендуется использовать двухфакторную аутентификацию для повышения безопасности учетных записей разработчиков.
        \item \textbf{Шифрование данных} используется для SSH-ключей и для HTTPS. Это помогает обеспечить защиту передаваемых данных от прослушивания и несанкционированного доступа даже в случае компрометации инфраструктуры.
        \item \textbf{Аудит действий} в GitHub: GitHub предоставляет подробные журналы аудита, которые регистрируют все действия, выполненные участниками проекта.
    \end{enumerate}
\end{frame}

\end{document}
